\documentclass{article}
\usepackage{listings, xcolor, graphicx, inconsolata, amsmath, amssymb}
\setlength{\parindent}{0pt}

\definecolor{lgray}{rgb}{0.95, 0.95, 0.95}
\definecolor{lgray2}{rgb}{0.75, 0.75, 0.75}

\title{MATH 330; Introduction}
\author{Gael Zarco}
\date{\today}

\begin{document}

\maketitle

% SECTION 1 %
\section{Notation} 
A number that belongs to the set of all real numbers, denoted as $\mathbb{R}$,
is referred to as a \textbf{Scalar} or a \textit{scalar value}.
\begin{itemize}
  \item Scalar values are always denoted using lower-case letters, i.e.,
  \textit{j, s, t, a, m}
    \begin{itemize}
      \item May also have subscripts which indicate that they are different
      numbers, i.e., \textit{$k_1, k_2, k_3$}.
    \end{itemize}
\end{itemize}

Notation for getting the sum of $k_1$ through $k_3$:
$$
  \sum_{i=1}^{3} k_i 
$$

$k \in \mathbb{R}$ means that $k$ belongs to the set of all real numbers.

\vspace{8pt}

The aboslute value of scalar, $k$, is denoted $|k|$, and is defined as:
$$
  |k| =
  \begin{cases}
    k,  & \text{if } k \ge 0,\\
    -k, & \text{if } k < 0,
  \end{cases}
$$
\begin{itemize}
  \item If the number is \textit{negative}, the minus sign is removed.
\end{itemize}

A \textbf{Set} is a small collection of, for example, integers. A set
containing the numbers 1, 2, 5 is denoted as $\{1, 2, 5\}$.
\begin{itemize}
  \item If it is desireable to have a variable, $i$, that can take on any number
    in the set, then it is denoted as: $ i \in \{1, 2, 5\}$.
\end{itemize}

Real numbers can take on values in a certain \textit{range}. For example:
\begin{itemize}
  \item If $x$ can take on any value from 0 to 1, inclusive: $ x \in [0, 1]$
  \item Parentheses denotes exclusive end points:
  $$
    x \in [-1, 2) \text{ denotes } -1 \le x < 2
  $$
\end{itemize}

% SECTION 2 %
\section{Trigonometry} 
SOHCAHTOA THROWBACK
\end{document}
