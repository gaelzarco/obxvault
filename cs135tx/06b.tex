\documentclass{article}
\usepackage{listings, xcolor, graphicx, inconsolata}
\setlength{\parindent}{0pt}

\definecolor{lgray}{rgb}{0.95, 0.95, 0.95}
\definecolor{lgray2}{rgb}{0.75, 0.75, 0.75}

\lstset{
  language=C++,
  aboveskip=10pt,
  belowskip=10pt,
  frame=leftline,
  xleftmargin=15pt,
  framexleftmargin=20pt,
  breaklines=true,
  basicstyle=\small,
  showstringspaces=false,
  backgroundcolor=\color{lgray},
  rulecolor=\color{lgray2},
  basicstyle=\ttfamily\footnotesize,
  keywordstyle=\color{blue},
  stringstyle=\color{red},
  commentstyle=\color{green},
  numbers=left,               
  numberstyle=\tiny\color{gray},
  stepnumber=1
}

\title{CS135; User-Defined Functions Cont}
\author{Gael Zarco}
\date{\today}

\begin{document}

\maketitle

% SECTION 1 %
\section{Void Functions}
\texttt{void} functions do not have a \texttt{functionType} and do not have any
\textit{formal parameters}.

\begin{lstlisting}[caption={Function Definition}]
  void functionName(formal parameter list) {
    statements
  }
\end{lstlisting}

User-defined void functions can be placed either before or after the function
\texttt{main}.
\begin{itemize}
  \item If placed after \texttt{main} $\rightarrow$ \textit{function prototype}
    required.
  \item Does not have a \textit{return type}.
    \begin{itemize}
      \item \texttt{return} statement without any value is typically used to
        exit the function early.
    \end{itemize}
\end{itemize}
\textbf{Coding Standard}: Functions are protoyped before \texttt{main} and
defined after \texttt{main}. Non-recursive functions should have
\textbf{ONE} \texttt{return} statement.

\subsection{Syntax}
\begin{lstlisting}[caption={Formal Parameter List Syntax}]
  dataType& variable, dataType& variable, ...
\end{lstlisting}

\texttt{\&} in the listing above means \textit{some} parameters will have
\texttt{\&} and some will not.

\begin{lstlisting}[caption={Function Call Syntax}]
  functionName(actual parameter list);
\end{lstlisting}

\begin{lstlisting}[caption={Actual Parameter List Syntax}]
  expression or variable, expression or variable, ...
\end{lstlisting}

\subsection{Parameter Types}
\begin{itemize}
  \item \textbf{Value Parameters}: Formal parameters that receive a copy of the
    content of the corresponding actual parameters.
  \item \textbf{Reference Parameters}: Formal parameters that receive a copy of
    the location (memory address) of the corresponding actual parameter.
\end{itemize}

When attaching \texttt{\&} after the \texttt{dataType} in the formal parameter list
of a function, the variable after that \texttt{dataType} becomes a reference
parameter.

% SECTION 2 %
\section{Value Parameters}
When function is called, for a value param, the actual value of the param is
copied into the corresponding formal param. Meaning, there is no connection
between the two as they are now separate entities. Therefore, the formal param
is manipulated with its own copy of the data.

% SECTION 3 %
\section{Reference Variables as Parameters}
Reference params receive the address (memory location) of the actual
param and enable direct manipulation of the actual param.

Useful in three situations, when:
\begin{itemize}
  \item actual param needs to be altered.
  \item you want to return more than one value from a func
    (\texttt{return} statements can return only \textbf{ONE} value).
  \item passing the address would save memory space and time relative to
    copying a large amount of data.
\end{itemize}

Remember to attach \texttt{\&} after the \texttt{dataType} in the formal param
list of a function to convert the following \texttt{variable} into a reference
param.

% SECTION 4 %
\section{Value and Reference Parameters and Memory Allocation}
\textbf{Local Variables} are vars declared in the body of the function and are
allocated in the function data area. During data manip, the address stored in
the formal param directs the computer to manip the data of the \textit{memory
cell} of that address (reference param). Value params possess a copy of the
actual param. Therefore, reference params can permanently change the value of
the actual param while value params cannot.

% SECTION 5 %
\section{Reference Params and Value-Returning Functions}
You can use reference params in value-returning functions, although it is not
recommended. Try converting it to a void function instead. If a function needs
to return more than one value, also convert it to a void function and use
appropriate reference params.

% SECTION 6 %
\section{Scope of An Identifier}
\textbf{Scope} of an identifier refers to where in the program an identifier is
accessible (identifier is the name of something in C++).

\begin{itemize}
  \item \textbf{Local Identifier}: Identifiers declared within a func/block
     $\rightarrow$ not accessible outside of the func/block.
  \item \textbf{Global Identifier}: Identifiers delcared outside of every func
    definition.
\end{itemize}

\textbf{C++ Does not allow nesting of functions}.

\vspace{8pt}
Rules when an identifier is accessed:
\begin{enumerate}
  \item Global identifierss accessible by func/block if:
  \begin{itemize}
    \item The identifier is declared before the func definition.
    \item The func name is different than the identifier.
    \item All func params have names different names than the identifier.
    \item All local identifiers (such as local vars) have names different than
      the name of the identifier.
  \end{itemize}
  \item \textbf{Nested Block} An identifier within a block is accessible:
  \begin{itemize}
    \item Only within the block and not at any point before or after.
    \item By blocks that are nested within that block, as long as the identifier
      does not share a name with the root block.
  \end{itemize}
\end{enumerate}

Notes about global variables:
\begin{enumerate}
  \item Some compilers initialize global vars to default values.
  \item \textbf{Scope Resolution Operator} in C++ is \texttt{::}
  \item By using scope resolution operator:
  \begin{itemize}
    \item A global var declared before the func/block definition can be accessed
      by the func/block even if the func/block has an identifier with the same
      name as the globar var.
  \end{itemize}
  \item To access a global var declared after the func definition, the func must
    not contain any identifier with the same name.
  \begin{itemize}
    \item Reserved word \texttt{extern} indicates that a global var has been
      declared elsewhere.
  \end{itemize}
\end{enumerate}

% SECTION 7 %
\section{Global Variables, Named Constants, and Side Effects}
A func that uses global variables is not independent.

\vspace{8pt}
If more than one func uses the same global var:
\begin{itemize}
  \item It can make it difficult to debug code.
  \item Problems caused in one area of the program can appear in a separate form
    in another.
\end{itemize}

Global named constants have no side effects.

\vspace{8pt}
\textbf{Coding Standard}: Global variables are prohibited at all times.
Global constants are acceptable.

% SECTION 8 %
\section{Static and Automatic Variables}
\begin{itemize}
  \item \textbf{Automatic Variable}: Memory is allocated at block entry and
    deallocated upon block exit.
  \begin{itemize}
    \item By default, variables declared within a block are automatic variables. 
  \end{itemize}
  \item \textbf{Static Variable}: Memory remains allocated as long as the
    program executes.
  \begin{itemize}
    \item Global variables declared outside of any block are \texttt{static}
      variables.
  \end{itemize}
\end{itemize}

Use reserved word \texttt{static} to delcare a \texttt{static} var.

\begin{lstlisting}[caption={\texttt{static} Variable Declaration}]
  static dataType identifier;
\end{lstlisting}

% SECTION 9 %
\section{Using Drivers and Stubs}
A \textbf{Driver} is a separate program used to test a function.

\vspace{8pt}
When results calculated by one func are needed in another, use a
\textbf{Function Stub}.
\begin{itemize}
  \item A func stub is a function that is not fully coded.
\end{itemize}

% SECTION 10 %
\section{Function Overloading}
In a C++ program, several functions can have the same name. \textbf{Function
Overloading} occurs when creating several functions with the same name.

\vspace{8pt}
Two funcs are said to have \textbf{different formal param lists} if
both funcs have either:
\begin{itemize}
  \item A different number of params.
  \item Conflicting \texttt{dataType} of formal params in atleast one instance.
\end{itemize}

Overloaded funcs must have different formal param lists. The \textbf{signature}
is the name and formal param list of the func. The param list supplied in a call
to an overloaded func determines which one is executed.

% SECTION 11 %
\section{Functions with Default Parameters}
In a func call, the number of actual and formal params must be the same.
\begin{itemize}
  \item C++ relaxed this condition for funcs with default params. 
  \item Can specify the value of a default param in the \textit{function
    prototype}. 
  \item If a value for a default param is not specified in a func call, the
    default value is used.
\end{itemize}

All default params must be the rightmost parameters of the func.
\begin{itemize}
  \item If default vaulue is not specified $\rightarrow$ You must omit all of
    the arguments to its right. 
\end{itemize}

Default values can be constants, global vars, or func calls. You cannot,
however, assign a constant value as a default value to a reference param.

\end{document}
