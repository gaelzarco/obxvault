\documentclass{article}
\usepackage{listings, color, xcolor, graphicx, inconsolata}
\setlength{\parindent}{0pt}

\definecolor{BlackText}{RGB}{110,107,94}
\definecolor{RedTypename}{RGB}{182,86,17}
\definecolor{GreenString}{RGB}{96,172,57}
\definecolor{PurpleKeyword}{RGB}{184,84,212}
\definecolor{GrayComment}{RGB}{170,170,170}
\definecolor{GoldDocumentation}{RGB}{180,165,45}
\definecolor{lgray}{rgb}{0.95, 0.95, 0.95}
\definecolor{lgray2}{rgb}{0.75, 0.75, 0.75}

\lstdefinelanguage{rust} {
    columns=fullflexible,
    keepspaces=true,
    frame=single,
    framesep=0pt,
    framerule=0pt,
    framexleftmargin=4pt,
    framexrightmargin=4pt,
    framextopmargin=5pt,
    framexbottommargin=3pt,
    xleftmargin=4pt,
    xrightmargin=4pt,
    basicstyle=\ttfamily\color{BlackText},
    keywords={
        true,false,
        unsafe,async,await,move,
        use,pub,crate,super,self,mod,
        struct,enum,fn,const,static,let,mut,ref,type,impl,dyn,trait,where,as,
        break,continue,if,else,while,for,loop,match,return,yield,in
    },
    keywordstyle=\color{PurpleKeyword},
    ndkeywords={
        bool,u8,u16,u32,u64,u128,i8,i16,i32,i64,i128,char,str,
        Self,Option,Some,None,Result,Ok,Err,String,Box,Vec,Rc,Arc,Cell,RefCell,HashMap,BTreeMap,
        macro_rules
    },
    ndkeywordstyle=\color{RedTypename},
    comment=[l][\color{GrayComment}\slshape]{//},
    morecomment=[s][\color{GrayComment}\slshape]{/*}{*/},
    morecomment=[l][\color{GoldDocumentation}\slshape]{///},
    morecomment=[s][\color{GoldDocumentation}\slshape]{/*!}{*/},
    morecomment=[l][\color{GoldDocumentation}\slshape]{//!},
    morecomment=[s][\color{RedTypename}]{\#![}{]},
    morecomment=[s][\color{RedTypename}]{\#[}{]},
    stringstyle=\color{GreenString},
    string=[b]"
}

\lstset{
  language=rust,
  aboveskip=10pt,
  belowskip=10pt,
  frame=leftline,
  xleftmargin=15pt,
  framexleftmargin=20pt,
  breaklines=true,
  basicstyle=\small,
  showstringspaces=false,
  backgroundcolor=\color{lgray},
  rulecolor=\color{lgray2},
  basicstyle=\ttfamily\footnotesize,
  keywordstyle=\color{blue},
  stringstyle=\color{red},
  commentstyle=\color{green},
  numbers=left,               
  numberstyle=\tiny\color{gray},
  stepnumber=1
}

\title{Template; Rust Programming Language}
\author{Gael Zarco}
\date{\today}

\begin{document}

\maketitle

A sample template for note-taking in Rust.

\section{Introduction and \texttt{main} function}

\begin{lstlisting}[caption={Example Program}]
fn main() {
    println!("Hello, World!");
}
\end{lstlisting}
\textit{Outputs "Hello World!" to the terminal.}

% \vspace{8pt}
% Title
% \begin{center}
%     \includegraphics[width=0.9\textwidth]{}
% \end{center}

\vspace{8pt}
The \textbf{Rust Prgramming Language} is a programming language with an emphasis on
memory-safety and speed.
\begin{itemize}
    \item Utilizes a paradigm known as \textbf{Ownership}.
    \item Memory managed by the \textbf{Borrow Checker}.
\end{itemize}

Always contains a \texttt{main} function.

\subsection{Subection}

\texttt{if program !contain main} $\rightarrow$ \texttt{error}.

\begin{lstlisting}[caption={\texttt{main} Function Syntax}]
fn main (..args) {
    statement(s)
}
\end{lstlisting}

\section{\textbf{Borrowing} and \textbf{Ownership}}

Rust's ownership system ensures memory safety without a garbage collector by enforcing strict rules on how values are accessed and modified.

\begin{itemize}
  \item \textbf{Ownership}: Every value in Rust has a single owner, and when the owner goes out of scope, the value is dropped.
  \item \textbf{Borrowing}: Allows references to use a value without taking ownership, preventing multiple mutable references at the same time.
  \item \textbf{Mutable vs Immutable Borrowing}: Multiple immutable references
  \texttt{\&T} are allowed, but only one mutable reference \texttt{\&mut T} can exist at a time.
  \item \textbf{Lifetimes}: Ensure borrowed references do not outlive the data they point to, preventing dangling references.
\end{itemize}

\section{Summary}

Rust is a powerful programming language with an emphasis on speed and memory
safety.

\end{document}
