\documentclass{article}
\usepackage{listings, xcolor, graphicx, inconsolata, amsmath}
\setlength{\parindent}{0pt}

\definecolor{lgray}{rgb}{0.95, 0.95, 0.95}
\definecolor{lgray2}{rgb}{0.75, 0.75, 0.75}

\lstset{
  language=C++,
  aboveskip=10pt,
  belowskip=10pt,
  frame=leftline,
  xleftmargin=15pt,
  framexleftmargin=20pt,
  breaklines=true,
  basicstyle=\small,
  showstringspaces=false,
  backgroundcolor=\color{lgray},
  rulecolor=\color{lgray2},
  basicstyle=\ttfamily\footnotesize,
  keywordstyle=\color{blue},
  stringstyle=\color{red},
  commentstyle=\color{green},
  numbers=left,               
  numberstyle=\tiny\color{gray},
  stepnumber=1
}

\title{MATH127; Formulae}
\author{Gael Zarco}
\date{\today}

\begin{document}

\maketitle

% SECTION 1 %
\section{Distance} 
\[
  d = \sqrt{(x_2 - x_1)^2 + (y_2 - y_1)^2}
\]

\section{Midpoint}
\[
  (x_m, y_m) = (\dfrac{x_1 + x_2}{2}, \dfrac{y_1 + y_2}{2}) 
\]

\section{Pythagorean Theorem}
\[
  a^2 + b^2 = c^2
\]
\textbf{Converse of Pythagorean Theorem} states that if the longest side of a
triangle equals the sum of the other 2 sides, it is a \textit{right} triangle.

\section{General Form}
\[
  ax + by = c
\]

\section{Slope}
\[
  \dfrac{\Delta y}{\Delta x} = \dfrac{y_2 - y_1}{x_2 - x_1}
\]

\section{Point-Slope}
\[
  y - y_1 = m(x - x_1)
\]

\section{Area of a Triangle}
\[
  A = \dfrac{1}{2}bh
\]

\section{Area of a Triangle}
\[
  A = \dfrac{1}{2}bh
\]

\section{General Form of a Circle}
\[
  Ax^2 + Ay^2 + Dx + Ey + F = 0
\]

\section{Standard Form of a Circle}
\[
  (x - h)^2 + (y - k)^2 = r^2
\]

\begin{itemize}
  \item Center: \texttt{(h, k)}. 
  \item If \texttt{h} or \texttt{k} are positive in \textbf{Standard Form} 
  $\rightarrow$ They reflect a negative value, respectively.
\end{itemize}

\section{Area of Square Inside Circle}
\[
  A = 2r^2
\]

\section{Average Rate of Change of Function Over Interval}
Average rate of change of $f(x)$ over interval $[a, b]$:

\[
  \dfrac{f(b) - f(a)}{b - a}
\]

This represents the slope of the \textbf{Secant Line} that connects the points
$(a, f(a))$ and $(b, f(b))$.

\end{document}
